%*******************************************************
% Abstract
%*******************************************************
%\renewcommand{\abstractname}{Abstract}
\pdfbookmark[1]{Abstract}{Abstract}
\begingroup
\let\clearpage\relax
\let\cleardoublepage\relax
\let\cleardoublepage\relax

\chapter*{Abstract}

Since their beginning in the late 1950s, the capabilities and applications of
\acp{AUV} have continuously evolved. Their most common applications include
imaging and inspecting different kinds of structures on the sea floor as well as
collecting oceanographic information: biological, chemical, and even
archaeological data.

Most of these applications require \textit{a priori} information of the area or
structure to be inspected, either to navigate at a safe and conservative
altitude or to pre-calculate a survey path. However there are other applications
where it's unlikely that such information is available (\eg exploring confined
natural environments like underwater caves). In these scenarios, \acp{AUV} must
operate in unexplored and cluttered environments, and therefore are more exposed
to collisions.

Although these \ac{AUV} applications share some common requirements with other
aerial and terrestrial robots (\eg localization, mapping, vision, etc.), they
are also different in significant ways. Navigating autonomously while conducting
these type of tasks in underwater environments demands taking into account
factors such as: the presence of external disturbances (currents), low-range
visibility and limited navigation accuracy. Dealing with such constraints
requires a path planner with online capabilities that can overcome the lack of
environment information and the global position inaccuracy, especially when
navigating in close proximity to nearby obstacles.

In this respect, this thesis presents an approach that endows an \ac{AUV} with the
capabilities to move through unexplored environments. To do so, it
proposes a computational framework for planning feasible and safe paths
online. This approach allows the vehicle to incrementally build a map of
the surroundings, while simultaneously (re)plan a feasible path to a
specified goal. To accomplish this, the framework takes into account motion constraints
in planning feasible 2D and 3D paths, \ie those that meet the vehicle's motion
capabilities. It also incorporates a risk function to avoid navigating close to
nearby obstacles.

To evaluate the proposed approach in different real-world scenarios, a series of
trials were conducted with the Sparus~II and the AsterX \acp{AUV},
torpedo-shaped vehicles that performed autonomous missions. These experiments
include simulated and in-water trials in different environments, such as
artificial marine structures, natural marine structures, and confined natural
environments.

\vfill

% \begin{otherlanguage}{ngerman}
% \pdfbookmark[1]{Zusammenfassung}{Zusammenfassung}
% \chapter*{Zusammenfassung}
% Kurze Zusammenfassung des Inhaltes in deutscher Sprache\dots 
% \end{otherlanguage}

\endgroup			

\vfill
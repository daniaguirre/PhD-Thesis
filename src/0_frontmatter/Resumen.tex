%*******************************************************
% Abstract
%*******************************************************
%\renewcommand{\abstractname}{Abstract}
\pdfbookmark[1]{Resumen}{Resumen}
\begingroup
\let\clearpage\relax
\let\cleardoublepage\relax
\let\cleardoublepage\relax

\chapter*{Resumen}

Desde sus inicios a finales de los años 50, las capacidades y aplicaciones de
los vehículos autónomos submarinos o AUVs (por sus siglas en inglés) han estado
bajo un continuo proceso de evolución. Las aplicaciones más comunes incluyen la
obtención de imágenes e inspeccción de diferentes tipos de estructuras, tales
como cascos de barcos, estructuras naturales en el fondo marino, así como la
recolección de información oceanográfica como datos biológicos, químicos e
incluso arqueológicos.

Muchas de estas aplicaciones requieren información \textit{a priori} del área o
estructura que va a ser inspeccionada, ya sea para navegar a una altitud segura
o para precalcular un camino para realizar estudios, el cual puede ser corregido
o modificado en tiempo real. Sin embargo, existen aplicaciones similares o nuevas,
como la exploración de entornos naturales confinados (\eg cuevas submarinas),
donde dicha información puede no estar disponible. En estos escenarios, los AUVs
debe operan en entornos desconocidos, y por lo tanto los AUVs están más
expuestos a colisiones.

Aunque estas aplicaciones de AUVs comparten algunos requiremientos comunes con
otras aplicaciones de robots aéreos y terrestres (\eg localización, mapeo,
visión, etc.), navegar autónomamente mientras se ejecutan este tipo de tareas en
entornos submarinos difiere en ciertos factores, como la presencia de
pertubaciones externas (corrientes), bajo rango de visibilidad y limitaciones en
la precisión del sistema de navegación. Para poder abordar dichas restricciones
se requiere un planificador de movimientos con capacidad de cómputo en tiempo
real, el que contribuya a superar las limitaciones en la información del entorno
y la falta de precisión de posicionamiento, en especial cuando se navega cerca
de los obstáculos de su alrededor.

En este sentido, esta tesis presenta una método para dotar un AUV con la
habilidad para moverse a través de entornos no explorados. Para ello, esta tesis
propone un método para calcular en tiempo real caminos factibles y seguros. El
método propuesto permite al vehículo construir incrementalmente un mapa del
entorno, y al mismo tiempo replaninficar el camino factible hacia la meta u
objetivo establecido. Para lograr esto, es necesario considerar las
restricciones de movimiento para planificar caminos 2D y 3D que sean factibles o
realizables.

Para evaluar el método propuesto, se realizaron diferentes experimentos con los
AUVs Sparus~II y AsterX, ambos vehículos tipo torpedo que realizaron
misiones de manera autónoma en diferentes escenarios. Estos experimentos
incluyen pruebas en simulación y en el agua en diferentes entornos, tales como
estructuras marinas artificiales, estructuras marinas naturales, así entornos
naturales confinados.

\vfill

% \begin{otherlanguage}{ngerman}
% \pdfbookmark[1]{Zusammenfassung}{Zusammenfassung}
% \chapter*{Zusammenfassung}
% Kurze Zusammenfassung des Inhaltes in deutscher Sprache\dots 
% \end{otherlanguage}

\endgroup			

\vfill
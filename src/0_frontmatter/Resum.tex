%*******************************************************
% Abstract
%*******************************************************
%\renewcommand{\abstractname}{Abstract}
\pdfbookmark[1]{Resum}{Resum}
\begingroup
\let\clearpage\relax
\let\cleardoublepage\relax
\let\cleardoublepage\relax

\chapter*{Resum}

Des dels inicis a finals dels anys 50, les capacitats i aplicacions dels
vehicles autònoms submarins o AUVs (per les seves sigles en anglès) han
experimentat un procés d'evolució continu. Les aplicacions més comunes són
l’obtenció d'imatges i inspecció de diferents tipus d'estructures, com per
exemple, cascos de vaixells o estructures naturals en el fons marí, i
l'adquisició d'informació oceanogràfica com dades biològiques, químiques i
arqueològiques.

Moltes d'aquestes aplicacions requereixen informació \textit{a priori} de l'àrea
o estructura que es vol inspeccionar, ja sigui per navegar a una altitud segura
o per calcular en avançat un camí que permeti realitzar els estudis, el qual pot
ser corregit o modificat a temps real. No obstant, existeixen aplicacions
similars o noves, com l'exploració d'entorns naturals confinats (\eg coves
submarines), on aquesta informació pot ser inexistent. En aquests casos, els
AUVs han d'operar en entorns desconeguts, pel que estan més exposats a
col·lisions.

Tot i que aquestes aplicacions de AUVs comparteixen alguns requeriments comuns
amb altres aplicacions de robots aeris i terrestres (\eg localització, mapeig,
visió, etc.), navegar autònomament al mateix temps que s'executen aquest tipus
de tasques en entorns submarins difereix en certs factors, com la presencia de
pertorbacions externes (corrents), baix rang de visibilitat i limitacions en la
precisió del sistema de navegació. Per poder tractar les esmentades restriccions
és necessari un planificador de moviments amb capacitat de processament en temps
real, el que ajudi a superar les limitacions en la informació de l'entorn i la
falta de precisió de posicionament, en especial quan es navega a prop dels
obstacles presents en el seu voltant.

En aquest sentit, aquesta tesis presenta una alternativa per dotar un AUV amb
l'habilitat de moure’s a través d'entorns no explorats. Per aconseguir aquesta
fita, aquesta tesis proposa un mètode per calcular en temps real camins
factibles i segurs. El mètode proposat permet al vehicle construir de forma
incremental un mapa de l'entorn, i al mateix temps replanificar un camí factible
cap a l'objectiu establert. Per assolir això, el mètode proposat te en compte
les restriccions de moviment del vehicle per planificar camins 2D i 3D que
siguin factibles o realitzables.

Per avaluar el mètode proposat, s'han realitzat diferents experiments amb els
AUVs Sparus~II i l'AsterX, els quals són vehicles de tipus torpede que van
realitzar missions de forma autònoma en diferents escenaris. Aquests experiments
inclouen proves en simulació i a l'aigua en diferents entorns, tal i com
estructures marines artificials, estructures marines naturals i entorns naturals
confinats.


\vfill

% \begin{otherlanguage}{ngerman}
% \pdfbookmark[1]{Zusammenfassung}{Zusammenfassung}
% \chapter*{Zusammenfassung}
% Kurze Zusammenfassung des Inhaltes in deutscher Sprache\dots 
% \end{otherlanguage}

\endgroup			

\vfill
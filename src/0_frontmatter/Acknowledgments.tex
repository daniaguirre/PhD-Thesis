%*******************************************************
% Acknowledgments
%*******************************************************
\pdfbookmark[1]{Acknowledgments}{acknowledgments}

% \begin{flushright}{\slshape    
%     We have seen that computer programming is an art, \\ 
%     because it applies accumulated knowledge to the world, \\ 
%     because it requires skill and ingenuity, and especially \\
%     because it produces objects of beauty.} \\ \medskip
%     --- \defcitealias{Carsten2006}{Donald E. Knuth}\citetalias{Carsten2006} \citep{Carsten2006}
% \end{flushright}



\bigskip

\begingroup
\let\clearpage\relax
\let\cleardoublepage\relax
\let\cleardoublepage\relax
\chapter*{Acknowledgments}

The work presented throughout this manuscript would not have been possible
without the help of some incredible people. Their continuous support, from the
beginning of this journey, till its end, helped me celebrate even the smallest
achievements, as well as overcome the difficult moments any PhD thesis involves.
I would like to give my gratitude to all of them.

En primer lugar me gustaría agradecer a mi familia en Armenia. A mi madre, por
buscar y querer de mi siempre lo mejor; por su carácter y decisión para afrontar
las dificultades de la vida; por sus sacrificios para ofrecerme las mejores
oportunidades. A Agustín Escobar, por hacer de mi la persona y el ingeniero que
soy, y por todos aquellos consejos que nunca olvidaré. Espero que este logro lo
estés disfrutando desde el infinito. A mi hermana Maria Paula, por su alegría y
su apoyo incondicional, por estar siempre ahí para mi sin importar la distancia.
A Denisa, quien también ya hace parte de esta familia, gracias por apoyarme en
los buenos y malos momentos de este doctorado, y por escuchar mis interminables
conversaciones sobre mis experimentos y artículos.

También me gustaría agradecer a mi familia en Neiva. A mi padre, por sus
llamadas y discusiones sobre la vida académica, por los buenos consejos y por
escucharme en los momentos que he tenido que tomar decisiones importantes. A mis
hermanos Daniela y Sergio Andrés, así como su madre Sonia Perdomo, gracias por
la buena energía, y los buenos deseos. Han sido cortos y pocos los momentos que
hemos podido compartir en los últimos años, pero gracias por estar siempre ahí.

Sin lugar a dudas, irme a Girona a hacer mi doctorado fue la mejor decisión
profesional y experiencia personal que pude tener. Girona siempre será como mi
segunda casa. Por esto me gustaría a agradecer a mi tutor, quien desde un inicio
me dio la posibilidad de preparar la propuesta de tesis con la que obtuve la
beca de Colciencias, propuesta que se ha convertido ahora en una realidad.
Gracias por el tiempo, las discusiones, las charlas de camino a Sant Feliu, y
por siempre ser crítico y realista a la hora de definir los objetivos. Al mismo
tiempo, me gustaría agradecer a Pere, por siempre estar allí para discutir temas
adicionales de la tesis, los experimentos y los artículos. Siempre fue para mi
como un co-tutor.

Un agradecimiento muy especial va para mis amigos y compañeros de trabajo en el
CIRS y el ViCOROB. Quisiera empezar con aquellos que estuvieron más cerca del
desarrollo de esta tesis. Gracias a Enric por ser como un tutor y mentor durante
mi primer año, por nuestras discusiones sobre path/motion planning, y por sus
consejos para escribir artículos. A Eduard, Èric y Narcís quisiera agradecerles
por siempre estar dispuestos a implementar, ajustar, mejorar la arquitectura del
robot, o simplemente acompañarme durante los experimentos en el mar. A Carles y
a Lluís, por tener los robots siempre a punto, solucionar los problemas
técnicos, y por darme siempre consejos prácticos de mis experimentos. Gracias a
mis compañeros de doctorado con los que compartí mis ``deadlines'', viajes y
competiciones, Arnau, Guillem, Albert, Jep, y Klemen. Gracias también a los
miembros seniors, Aggelos, Tali (en especial por darle revisiones adicionales a
mis papers, abstracts, y pósters), Ricard, David y Nuno. Quisiera agradecer
además al área administrativa del ViCOROB. Joseta, Mireia, Anna, Olga y Bego,
que siempre solucionaron todos los problemas con la Universidad, las
conferencias, los viajes y demás.

I would also like to thank my foreigner friends who have been part of ViCOROB.
Francesco, Mojdeh and Guillaume, Sonia, Habib (and Sarah), Konstatin (and
Amanda), Shihav, and Sharad- I will never forget our amazing multicultural
dinners. I learned so much from all of you. And since the group was
ever-changing, also its latest members: Dina, Khadidja, Klemen, Richa, and
Patryk.

One of the most important and decisive experiences during my PhD was without a
doubt my visit to Rice University in Houston. I would especially like to thank
Lydia Kavraki and Mark Moll, my advisors there. Their time, dedication and
contributions to this thesis are invaluable. I would also like to thank all the
people at the Kavraki Lab: Morteza, Ryan, Didier, Dinler, Jayvee, Sarah, and
Keliang. Your advice and comments during our meetings not only helped me in my
work but also made my stay there an incredible experience.

Por último me gustaría agradecer a las fuentes de financiamiento de mis
estudios, experimentos y publicaciones. Al Departamento Administrativo de
Ciencia, Tecnología e Innovación (Colciencias), de Colombia, que me otorgó la
beca de estudios doctorales. También me gustaría agradecer a la Unión Europea, a
la Generalitat de Catalunya y al Gobierno Español, que por medio de proyectos
financiaron parte de mi investigación.



% Put your acknowledgments here.
% 
% Many thanks to everybody who already sent me a postcard!
% 
% Regarding the typography and other help, many thanks go to Marco 
% Kuhlmann, Philipp Lehman, Lothar Schlesier, Jim Young, Lorenzo 
% Pantieri and Enrico Gregorio\footnote{Members of GuIT (Gruppo 
% Italiano Utilizzatori di \TeX\ e \LaTeX )}, J\"org Sommer, 
% Joachim K\"ostler, Daniel Gottschlag, Denis Aydin, Paride 
% Legovini, Steffen Prochnow, Nicolas Repp, Hinrich Harms, 
%  Roland Winkler, Jörg Weber, Henri Menke, Claus Lahiri, 
%  Clemens Niederberger, Stefano Bragaglia, Jörn Hees, 
%  and the whole \LaTeX-community for support, ideas and 
%  some great software.
% 
% \bigskip
% 
% \noindent\emph{Regarding \mLyX}: The \mLyX\ port was intially done by 
% \emph{Nicholas Mariette} in March 2009 and continued by 
% \emph{Ivo Pletikosi\'c} in 2011. Thank you very much for your 
% work and for the contributions to the original style.


\endgroup




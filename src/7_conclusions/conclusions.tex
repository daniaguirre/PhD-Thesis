% this file is called up by thesis.tex
% content in this file will be fed into the main document

\chapter{Conclusions}\label{ch:conclusions} % top level followed by section,
% subsection


% ----------------------- paths to graphics ------------------------

% change according to folder and file names
\ifpdf
    \graphicspath{{7_conclusions/figures/PNG/}{7_conclusions/figures/PDF/}{7_conclusions/figures/}}
\else
    \graphicspath{{7_conclusions/figures/EPS/}{7_conclusions/figures/}}
\fi


% ----------------------- contents from here ------------------------


\section{Summary of Completed Work}

This thesis has addressed the problem of planning paths online for an \acf{AUV},
which navigates under motion constraints through unexplored environments. This
kind of missions require the vehicle to incrementally map the surroundings,
while simultaneously replanning a feasible and safe path. After providing an
overview of the general problem and the proposed objectives to tackle it,
Chapter~\ref{ch:state_of_the_art} presented the state of the art of path/motion
planning for robotic systems. This review started with discussing the general
planing problem, then it classified the available techniques, and finally it
discussed their most common extensions. For this latter part, the review put
special emphasis on those methods that have been used with underwater vehicles.

Chapter~\ref{ch:motion_constratins} presented the use of sampling-based methods
to plan feasible and constant-depth paths for a torpedo-shaped \ac{AUV}.
Here the term feasible refers to paths that take into consideration the
vehicle's limitations, either kinematic or dynamic ones. In the case analysed in
this thesis, torpedo-shaped \acp{AUV} are kinematically constrained, since they
cannot conduct pure lateral motion. Instead, they are required to move forward
or backward while conducting turning maneuvers. Two different approaches to
consider such limitations were presented and discussed. The first one directly
uses the \ac{AUV} kinematic equation of motion. The second one employs Dubins
curves to characterize the \ac{AUV} motion with straight line segments and
circular arcs. Both approaches were evaluated and compared in different
simulations. This allowed identifying their advantages and drawbacks, which led
to establish the one using Dubins curves as the best approach.

Chapter~\ref{ch:planning_3D} proposed an alternative formulation to plan
feasible and variable-depth paths for a torpedo-shaped \ac{AUV}. This
formulation extended the Dubins curves by including the \ac{AUV} vertical
motion, and it was also used with a sampling-based planning method. This sought
to define a general strategy to plan \ac{3D} paths, which took into account both
the vehicle's lateral and vertical motion constraints. This approach was
evaluated in different simulated scenarios, where it was compared with another
approach that does not include the vehicle's motion constraints involved.
Results proved that the proposed approach allows an \ac{AUV} to follow more
accurately \ac{3D} paths.

While simulation tests in Chapters~\ref{ch:motion_constratins} and
\ref{ch:planning_3D} assumed fully mapped environments, the main objective of
this thesis was to endow an \ac{AUV} with the capability to move through
unexplored environments. In order to do so, Chapter~\ref{ch:plann_online}
introduced an online mapping and path/motion planning framework for \acp{AUV}.
The framework uses an Octomap to incrementally build a representation of the
surroundings. To calculate the \ac{AUV} path, it includes a sampling-based
planner that employs the extended Dubins curves formulation for dealing with
\ac{2D} and \ac{3D} missions. Furthermore, the planner not only computes a
feasible path, but also attempts to obtain a safe one by minimizing its
associated risk. Finally, the framework incorporates new strategies that seek to
reduce the running time, which is a critical requirement for the intended
applications and their online computation limitations. The framework was
evaluated with simulations in different scenarios.

In order to completely validate the proposed framework and its properties,
Chapter~\ref{ch:applications} presented different experiments that were
conducted by torpedo-shaped \acp{AUV}. This included four different real-world
scenarios:

\begin{inparaenum}[1)]
\item \textit{Planning constant-depth paths to move through artificial marine
structures}. In this case, the Sparus~II \ac{AUV} traversed multiple times a
breakwater structure, which is composed of a series of concrete blocks separated
by four-meter gaps. At the beginning of the mission, the vehicle was not
provided with a map of the surroundings. This required the vehicle to
incrementally build a map, while planning a feasible and safe path to the
different specified goals. Furthermore, the vehicle was equipped with optical
cameras to gather images, which were used to create a \ac{3D} reconstruction of
the traveled area.

\item \textit{Planning constant-depth paths to move through natural marine
structures.} In this case, the Sparus~II navigated through a natural underwater
canyon made by rocky formations. This experiment sought to prove the
framework's capabilities in a more challenging scenario with non-regular shape
obstacles. As occurred in the previous experiment, the vehicle was not provided
with an initial map of the area, and it was also equipped with optical cameras.
The result of the mission also included a photo-realistic \ac{3D} reconstruction
of the natural scenario.

\item \textit{Planning variable-depth paths to move through confined marine
environments.} In the previously mentioned experiments, the vehicle navigated at
a constant depth due to a sensor limitation. However, in order to validate the
capability to plan \ac{3D} paths, a simulated Sparus~II conducted missions over
a real-world dataset of a cave complex. This scenario required solving
consecutive start-to-goal queries of different depths. As occurred with the
previous experiments, the vehicle was not provided with an initial map of the
surroundings.

\item \textit{The autonomous survey replanning for gap filling and target
inspection.} In this case, the AsterX \ac{AUV} conducted an inspection of an
area of interest. The mission was composed of two phases. The first one required
the \ac{AUV} to follow a preplanned coverage path. The second one guided the
vehicle to further inspect either potential targets or gaps (\ie areas not
correctly covered). For this latter phase, the extended Dubins curves
formulation was used to plan \ac{3D} paths to guide the \ac{AUV} closer from the
potential targets. This experiment sought to complement the validation of
planning \ac{3D} paths with a real-world vehicle.
\end{inparaenum}

These tests allowed proving the viability of the proposed approach under
real-world conditions. This included the assessment of the computational
efficiency, in which a single embedded computer was capable of conducting
simultaneously mapping, planning, and control tasks. Furthermore, along the
development of the proposed framework, the success rate increased from $10-20\%$
during the initial tests, to $80-90\%$ when using the final version of the
framework. However, it is also important to mention that those failing cases
were mainly due to sensor limitations.

\section{Review of Contributions}

Aiming to endow \acp{AUV} with the capabilities required to operate in
unexplored environments, this thesis has proposed an online mapping and
path/motion planning framework. This led to extend and develop different
strategies that contribute to the current state-of-the-art for underwater
vehicles. Such contributions have been presented and peer-reviewed along
different~\nameref{sec:publications}, and they can be gathered in four main
aspects:

\begin{description}

\item[Path/Motion Planning online] This thesis established that in order to move
through unexplored environments, an \ac{AUV} has to be capable of incrementally
mapping the surroundings, while simultaneously planning collision-free paths.
However, an initial approach in this thesis was to have two vehicles to conduct
missions in unexplored environments. The first vehicle had to navigate at a safe
altitude, and it was assumed to be equipped with a downward-looking multibeam
sonar that allowed to build online a map. Such a map was used to simultaneously
plan collision-free paths for a second vehicle that navigated in close-proximity
to the sea bottom~[NGCUV'15]. Although the previous approach was limited to
simulation tests, it did contribute to define a first mechanism for simultaneous
mapping and planning~[OCEANS'15]. This mechanism later became the framework
thoroughly explained in Chapter~\ref{ch:plann_online}. In its first version, the
framework used an anytime tree-pruning strategy that \textit{opportunistically
check} states for collision~[ICRA'15], but it was later improved by
\textit{reusing the last best known solution}~[IROS'16].

\item[Planning feasible motion for AUVs] Although a simultaneous mapping and
planning mechanism allowed an \ac{AUV} to move through an unexplored
environment~[ICRA'15], results also showed multiple replanning maneuvers.
The main reason was that the \ac{AUV} was not capable of accurately following
the provided paths. This thesis identified the necessity to plan not only
collision-free paths, but also feasible ones. This means that the calculated
paths must take into consideration the vehicle's motion constraints, thus
minimizing unexpected vehicle's trajectories when attempting to follow the
calculated path. This thesis proposed and validated the use of Dubins curves to
characterize the constant-depth \ac{AUV} motions~[IROS'16], which was later
extended to consider full \ac{3D} trajectories~[JFR'17].

\item[Planning safe paths] \acp{AUV} operate in complex environments, where
they can be affected by external perturbations such as waves and currents.
Furthermore, \acp{AUV}' navigation system has a position error that cannot be
corrected while submerged, especially in environments where a mother ship
cannot be used with an \ac{USBL} system. These operation conditions can lead the
vehicle to risky situations, especially when moving in close-proximity to nearby
obstacles. This thesis evaluated different alternatives to plan not only
collision-free and feasible paths, but also ones that attempt to minimize the
risk of collision. A first alternative was to maintain a safe distance from the
surroundings~[NGCUV'15]. However, it was later replaced with an optimization
function that combines the length and the risk associated with a calculated
path~[IROS'16].

\item[Experimental Evaluation] This thesis has extensively proved the proposed
framework and its newly introduced capabilities. This validation mainly involved
the Sparus~II \ac{AUV}. The experiments included simulated and in-water trials
in different scenarios, such as artificial marine structures
(breakwater)~[ICRA'15, IROS'16], natural marine structures (underwater
canyon)~[SENSORS'16], confined natural environments (caves complex)~[JFR'17].
Furthermore, the capability of planning feasible \ac{3D} paths was validated
with the AsterX \ac{AUV}~[OCEANS'17].

\end{description}

\section{Future Work}

This thesis cannot be considered a final and definitive solution for the problem
of planning feasible and safe paths for \acp{AUV}. However, it does contribute a
further step towards better and more reliable underwater vehicles. In doing so,
this thesis has established the basis for challenging future work that will
continue extending the \acp{AUV}' capabilities.

\begin{description}
\item[In-water Trials for 3D Mapping and Planning] Even though the framework
proposed in this thesis supports both \ac{2D} and \ac{3D} missions, its
capability of mapping \ac{3D} environments was only evaluated with simulations
over virtual and real-world datasets. This was mainly due to technical
limitations, such as the absence of a mechanism that allowed the \ac{AUV} to
rotate a forward-looking multibeam sonar. Therefore, the next and immediate step
to continue this work could be to conduct in-water trials in \ac{3D} scenarios,
such as the caves complex mentioned in Section~\ref{sec:caves_experiments}.

\item[Planning Varying Speed Motions] The framework proposed in this thesis uses
a sampling-based method with Dubins curves as a steering function. This
formulation assumes that the \ac{AUV} moves with a constant surge speed and a
maximum turning rate. In this way, the \ac{AUV} paths can be parameterized with
straight line segments and circular arcs of constant radius. However, there are
situations in which the vehicle should be capable of conducting tight turns.
Some of these situations where presented in Section~\ref{sec:caves_experiments},
where the \ac{AUV} surge speed was reduced to allow the planner finding a way
out of narrow tunnels. Therefore, another possible extension for future work is
to analyse alternatives to plan feasible and safe paths with varying speeds, thus
permitting maneuvers with different turning radius.

\item[View/Inspection Planning] As it was mentioned when discussing the
experiments presented in Chapter~\ref{ch:applications}, this thesis did not have
as an objective to provide a strategy to inspect an area or structure of
interest. Another possible future work from this thesis can be the development
of methods to efficiently conduct such inspections. In this case, the proposed
framework can be used as a low-level layer for a high-level control pipeline.
This latter will have to establish the trajectory that \ac{AUV} should follow in
order to fully inspect an area or structure. As an example of this possibility,
some of the characteristics of the proposed framework have been already used in
a view planning framework for \acp{AUV}~[RA-LETTERS'17].

\item[Path/Motion Planning under Uncertainty] One of the requirements identified
for moving through unexplored environments was to attempt minimizing the risk of
collision. Having this mind, this thesis presented an optimization function that
allows combining the length and the risk associated with the path. This approach,
however, is heuristically established. Another possible future work is to
propose an alternative methodology to calculate safe paths. Such a new approach
should consider the different sources of uncertainty to establish the validity
of the vehicle's states. In this case, the uncertainty might include the one
associated with the model, the controller, the navigation system, and/or the
exteroceptive sensors that detect nearby objects. Most of the current work on
this matter is still computationally expensive, although there are some
promising formulations that could be extended for \acp{AUV}. Some these
alternatives were analysed during the development of this work, and they are
currently under studied as an ongoing work at University of Girona~[ICRA'18].

\end{description}

% ---------------------------------------------------------------------------